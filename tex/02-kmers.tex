\chapter{K-mer counting}

% intor o kmeroch a podobne

Counting of the \kmer (substring of length $k$) occurences is one of the key problem in many applications in Bioinformatics.
It is used in error correction algorithms, DNA assemblers, sequence aligners and repeat detectors. It is also suitable for various sequence reeds analysis and genome size estimation.

In NGS data, there are usually a lot of reads, containing a lot of errors implying a large amount of unique \kmers.
This causes a high memory usage of naive approach to this problem.
\todo{mozno nejaky priklad s cislami}

There are several existing tools addressing this issue for \kmer counting available, for example tallymer\cite{tallymer}, jellyfish\cite{jellyfish}, bfcounter\cite{bfcounter}, dsk\cite{dsk} and khmer\cite{khmer}. Each of the tool must handle some trade of between speed, memory usage, disk usage, precision and random access capability.

There are multiple approaches to the \kmer counting problem. Most of them involve hash tables, bloom filters or suffix arrays.
\todo{strucne popisat niektore zo spomenutych DŠ}

\section{Naive solution}

Let us first briefly explain the naive solution, so we can extend and compare it to the methods used in previously mentioned tools.

Suppose we want to count \kmers in sequence $S$.
We create a hash table $T$, and then go through the sequence $S$.
At each position $i$, we take the \kmer $K_i$ corresponding to that  position, compute $hash(K_i)$ and increase the corresponding item\footnote{Suppose we have some collision resolution algorithm.} in the table.

We can then query the hash table for the number of occurrences of particular \kmer, or iterate through it to get all the counts.


% cosi o pouziti a pod
