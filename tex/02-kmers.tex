\chapter{K-mer counting}

% intor o kmeroch a podobne

% toto mam z jellyfishu
Counting the number of occurrences of every \kmer (substring of length k) in a long string is a central subproblem in many applications, including genome assembly, error correction of sequencing reads, fast multiple sequence alignment and repeat detection.

In this chapter, we intorduce several approaches to \kmer counting problem.
% Recently, the deep sequence coverage generated by next-generation sequencing technologies has caused the amount of sequence to be processed during a genome project to grow rapidly.

\section{Algorithms}

Firstly, we present several algorithms used in \kmer counting.

\subsection{Burrows-Wheeler Transform}

The Burrows-Wheeler Transform (BWT) is a way of permuting the characters of a string $T$ into
another string $BWT(T)$. This permutation is reversible; one procedure exists for turning $T$ into
$BWT(T)$ and another exists for turning $BWT(T)$ back into $T$. The transformation was originally
discovered by David Wheeler in 1983, and was published by Michael Burrows and David Wheeler
in 1994 \cite{burrows1994block}.

The BWT has two main applications: compression and indexing.
\cite{langmead2013bwt}

$T$ denotes the string we would like to transform, and $m = |T|$ (the length of $T$). We assume that $T$ ends with a terminator character, denoted \$. We define \$ to be a character that does not appear elsewhere in $T$, and which is lexicographically prior to all other characters. In the case of DNA
strings, for example, the alphabet order with \$ might be \$ < A < C < G < T.
Take $T = abaaba\$$. First, we write down the rotations of $T$: the distinct strings we can make from T by repeatedly taking a character from one end and sticking it on the other:

\begin{array}{ccccccc}
\$ & a & b & a & a & b & a\\
a & \$ & a & b & a & a & b\\
b & a & \$ & a & b & a & a\\
a & b & a & \$ & a & b & a\\
a & a & b & a & \$ & a & b\\
b & a & a & b & a & \$ & a\\
a & b & a & a & b & a & \$
\end{array}

By writing them stacked vertically, we’ve created an m × m matrix. Now we sort the rows of
the matrix lexicographically (i.e. alphabetically):
$ a b a a b a
a $ a b a a b
a a b a $ a b
a b a $ a b a
a b a a b a $
b a $ a b a a
b a a b a $ a
1
This is the Burrows-Wheeler Matrix (BWM(T)). The final column of BWM(T), read from
top to bottom, is BW T(T). So for T = abaaba$, BW T(T) = abba$aa.
\subsection{Bloom Filters}

\subsection{Hashing}

% cosi o pouziti a pod
