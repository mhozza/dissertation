\chapter{Repeats and Gene Families}\label{chap:repeatsfamilies}

\todo{short general intro, what do they have in common, maybe definitions and description how are they created...}

\section{Repeats}

\todo{intro to repeats - depends on what will be in general intro}

\todo{clanky:}
\begin{itemize}
  \item !important \cite{novak2010graph}
  \item !important \cite{pertea2003tigr}
  \item !important \cite{swaminathan2007global}
  \item \cite{gu2008identification}
  % \item \cite{sveinsson2013transposon} - blbost
  \item \cite{shapiro2005repetitive}
\end{itemize}

\todo{conclusion}

% Velmi zhruba o com to ma asi byt
% - nejake problemy a ako ich riesit - dat je vela, da sa spravit nizsie pokrytie (aj random samplovanim)
% - vlastnosti repeatov - je ich viac ako beznych sekvencie - teda ked nahodne sekvenujeme, tak ocakavame ze tam bude viac podobnych sekvencii ktore zodpovedaju repeatu ako podobnych co zodpovedaju nerepeatu (repeaty maju ako keby vacsi coverage)
% - ked znizujeme pokrytie, znizujeme pravdepodobnost ze sa dany kus sekvencii vyskytne v datach. kedze repeatov je viac, tak tie maju vyssiu pravdepodobnost ze sa dostanu do dat, cize vieme pokrytie nastavit tak, aby to co ostane v datach boli s vysokou pravdepodobostov repeaty, a s vysokou pravdepodobnostou sa tam dostanu vsetky (resp ocakavame ze sa ich tam dostane dost velka cast)
% - niekolko metod -> nastavit pokrytie dost nizko a detegovat komponenty suvislosti (tclust, nastudovat)
% - urobit ciastocne assembly a zobrat kontigy (tiez nastudovat)
% - grafove klastrovanie\cite{novak2010graph} -> pisu ze celkom sikovne
% - spomenut rozdieli oproti modelovaniu repeatov z predoslej kapitoly

\section{Gene Families}
\todo{intro to gene families - depends on what will be in general intro}

\todo{clanky:}
\begin{itemize}
  \item \cite{hahn2005estimating}
\end{itemize}

\todo{conclusion}

\section{Conclusion}
\todo{conclusion}
