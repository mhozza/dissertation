\chapter{Genome Size and Coverage Estimation}

The \emph{coverage} of genome by the sequencing data is the average number of sequencing reads covering each base. If the genome size is known, the coverage can be computed as a ratio of sequencing data size (the sum of lengths of all reads in sequencing data) and the genome size. Thus, the estimation of the genome size and the coverage are related problems.

The estimate of genome size can be very useful in experiment design in biology. If it is possible to estimate the genome size from low coverage sequencing data, biologists can do this estimate with low cost prior the experiment and use the information to better design the experiment.
The estimate of the coverage can be used in various sequencing data analysis, e.g.\ repeat analysis and analysis of gene families (see chapter~\ref{chap:repeatsfamilies}). \todo{pozri do clankov aku maju oni motivaciu}

One of possible approaches to estimate the genome size is assembling the genome and computing the size of assembled genome directly. However, this approach has several issues. The genome assembly is very hard problem. In most cases we cannot assemble the genome only from NGS data. Even if we are able to build a partial assembly, there are usually assembling errors around repetitive sequences, which can be very long and can has significant influence on the genome size. In addition, to do a genome assembly, we need to have high coverage data (which is more expensive and impossible for very large genomes).
The better approach is to estimate the genome size from the data directly. This can be done statistically by looking at sequence overlaps. \todo{pozriet do clankov, ci tam este neico nie je}

\section{Using K-mer Histograms for Genome Size Computation}

%cosi o tom ze pozerat na overlapy je narocne a kmery su na to fajne

% \section{Williams}
%
% \section{Watermann}
%
% \section{Covest}

\section{Conclusion}
\todo{conclusion}
