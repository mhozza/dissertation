% Zaoberáme sa dvomi dôležitými bioinformatickými problémami: anotáciou sekvencií
% a zarovnávaním sekvencií. V práci sa sústredíme na využitie skrytých Markovových modelov (HMM),
% dobre známych generatívnych pravdepodobnostných modelov.
%
% V prvej časti študujeme anotáciu sekvencií, konkrétne dvojstupňové dekódovacie algoritmy a
% výpočtové problémy, ktoré s nimi súvisia. Ukážeme, že dvojstupňové algoritmy
% môžu zlepšiť presnosť dekódovania a dokážeme, že tri problémy vhodné pre prvý stupeň výpočtu sú NP-ťažké:
% problém najpravdepodobnejšej množiny,
% problém najpravdepodobnejšej reštrikcie a problém najpravdepodobnejšej stopy.
%
% Druhá časť sa zaoberá zarovnávaním sekvencií, ktoré obsahujú tandemové
% opakovania. Tandemové opakovania sú  opakujúce sa časti genomických sekvencií,
% ktoré často spôsobujú chyby v zarovnaniach. Aby sme vyriešili tento problém, vyvinuli
% sme nový HMM, ktorý modeluje zarovnania obsahujúce tandemové opakovania a
% skombinovali sme ho s existujúcimi ako aj novými dekódovacími algoritmami. Náš
% prístup sme vyhodnotili experimentálne.
%
% V oboch problémoch sme používali dekódovacie algoritmy na zlepšenie presnosti
% predikcií HMM. Dekódovacie algoritmy sú často podceňované  a väčšina vývoja
% ide do vytvárania topológie HMM. Avšak správnym výberom dekódovacej metódy môžeme
% dosiahnuť významné zlepšenie predikcií.
